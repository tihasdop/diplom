\documentclass[14pt]{extarticle} %Класс позволяет использовать базовые шрифты бОльших размеров
\usepackage[utf8x]{inputenc} %кодировка файла макета utf8
\usepackage[russian]{babel} 
\usepackage[left=25mm,right=15mm,top=12mm,bottom=20mm]{geometry} %Попытка разобраться с полями страниц
\usepackage{ntheorem} %окружение для настройки теорем 
\usepackage{graphicx} %работа с рисунками
\usepackage[labelsep=period,figurewithin=none,tablewithin=none]{caption} %подписи к объектам (рисунки, таблицы)
\usepackage{listings} %работа с листингами
\usepackage{indentfirst} %отступ первого абзаца в разделе
\usepackage{enumitem} %настройка маркированных и нумерованных списков (см. примеры настройки в тексте)
\usepackage{url} %формирование ссылок на электронные источники
\usepackage{fancyhdr} %Настройка нумерации страниц
\usepackage{tocloft} %Настройка заголовка для содержания

%====================================================================
%Настройки макета
%----------------
%Содержимое этого блока не должно подвергаться изменению
%====================================================================
\selectlanguage{russian}
\setlength{\parindent}{1.25cm}

%---------Настройка подписей к таблицам
\DeclareCaptionFormat{mplain}{#1#2\par \centering #3\par}
\captionsetup[table]{format=mplain,
justification=raggedleft,%
labelsep=none,%
singlelinecheck=false,%
skip=3pt}

%---------Настройка подписей к таблицам
\captionsetup{figurename=Рисунок}

%Настройка нумерации страниц
\fancyhf{} % clear all header and footers
\renewcommand{\headrulewidth}{0pt} % remove the header rule
\rfoot{\small \thepage}
\pagestyle{fancy}

%Настройка заголовка для содержания
\renewcommand{\cfttoctitlefont}{\hfill\normalfont\large\bfseries}
\renewcommand{\cftaftertoctitle}{\hfill\thispagestyle{empty}} 

%Настройка теорем
\theoremseparator{.}

%---------Команды рубрикации--------------

%Заголовки
\makeatletter
\renewcommand{\section}{\@startsection{section}{1}%
{\parindent}{-3.5ex plus -1ex minus -.2ex}%
{2.3ex plus.2ex}{\normalfont\large\bfseries}}

\renewcommand{\subsection}{\@startsection{subsection}{2}%
{\parindent}{-3.5ex plus -1ex minus -.2ex}%
{1.5ex plus.2ex}{\normalfont\large\bfseries}}

\renewcommand{\subsubsection}{\@startsection{subsubsection}{3}%
{\parindent}{-1.5ex plus -1ex minus -.2ex}%
{0.5ex plus.2ex}{\normalfont\bfseries}}
\makeatother

%Команда уровня главы
\newcommand{\mysection}[1]{
 \newpage
 \refstepcounter{section}
 {
  \section*{Глава \thesection. #1 \raggedright }
 }
 \addcontentsline{toc}{section}{Глава \thesection. #1} 
}

%Команда уровня параграфа
\newcommand{\mysubsection}[1]{
 \refstepcounter{subsection}
 \subsection*{\thesubsection. #1 \raggedright}
 \addcontentsline{toc}{subsection}{\thesubsection. #1}
}

%Команда третьего уровня
\newcommand{\mysubsubsection}[1]{
\refstepcounter{subsubsection}
% \addcontentsline{toc}{subsubsection}{\thesubsubsection. #1}
\subsubsection*{#1 \raggedright}
}

%Оформление Приложений
\newcounter{appendix}
\newcommand{\addappendix}[1]{
 \newpage
 \refstepcounter{appendix} 
 \section*{ПРИЛОЖЕНИЕ \theappendix. \\#1}
 \addcontentsline{toc}{section}{ПРИЛОЖЕНИЕ \theappendix. #1}
}

%Команда ненумерованной главы
\newcommand{\mynonumbersection}[1]{
\newpage
{
	\centering\section*{#1}
}
\addcontentsline{toc}{section}{#1} 
}

%--------Настройка маркированных и нумерованных списков
\setlist{itemsep=0pt,topsep=0pt}

%--------Настройка листингов программного кода
\lstloadlanguages{C,[ANSI]C++}%!настройка листинга
%Можно подключить другие языки (см документацию к пакету)

%--------Тонкая настройка листингов
\lstset{
inputencoding=utf8x,
extendedchars=false,
showstringspaces=false,
showspaces=false,
keepspaces = true,
basicstyle=\small\ttfamily,
keywordstyle=\bfseries,
tabsize=2,                      % sets default tabsize to 2 spaces
captionpos=t,                   % sets the caption-position to bottom
breaklines=true,                % sets automatic line breaking
breakatwhitespace=true,        % sets if automatic breaks should only happen at whitespace
title=\lstname,                 % show the filename of files included with \lstinputlisting;
basewidth={0.5em,0.45em},
}

%----------Настройка подписей к листингам
\renewcommand{\lstlistingname}{Листинг}

%------------Подключение стиля для оформления списка литературы
\makeatletter
\renewcommand{\@biblabel}[1]{#1.\hfill}
\makeatother
\bibliographystyle{ugost2003s}
\PrerenderUnicode{ЙЦУКЕНГШЩЗХЪЭЖДЛОРПАВЫФЯЧСМИТЬБЮйцукенгшщзхъэждлорпавыфячсмитьбю}

%-----------Формирование подписей к объектам с подчинением главе
%\renewcommand{\thefigure}{\arabic{section}.\arabic{figure}}
%\renewcommand{\thetable}{\arabic{section}.\arabic{table}}
%====================================================================
%Настройки макета
%----------------
%Содержимое предыдущего блока не должно подвергаться изменению
%====================================================================

%=============================
%Персональная настройка макета
%=============================
%!!!
%Здесь могут располагаться дополнительные команды для персональной тонкой настройки


%=============================
%Конец Персональная настройка макета
%=============================
\begin{document}
	
%%%--------Титульная страница
%==Титульная страница
\thispagestyle{empty}
\begin{center}
	Министерство образования и науки Российской Федерации\\
	Федеральное государственное бюджетное образовательное\\
	учреждение высшего образования\\
	<<Иркутский государственный университет>>\\
	(ФГБОУ ВО <<ИГУ>>)\\
	Институт математики, экономики и информатики\\
	Кафедра информационных технологий\\
\end{center}

\vspace{2.7cm}

\begin{center}
	{\bf 
		ВЫПУСКНАЯ КВАЛИФИКАЦИОННАЯ РАБОТА
		БАКАЛАВРА\\[1mm]
		по направлению <<02.03.03 Математическое обеспечение и \\[1mm] 
		администрирование информационных систем>>\\[1mm]
%		профиль подготовки <<Информатика и компьютерные науки>>
	}  
	
	\vspace{0.9cm}
	
	{
		РАЗРАБОТКА ИНТЕРПРЕТАТОРА ФУНКЦИОНАЛЬНОГО ЯЗЫКА
		ПРОГРАММИРОВАНИЯ HOPE ДЛЯ ОПЕРАЦИОННОЙ СИСТЕМЫ ANDROID
	} %Текст должен быть набран ЗАГЛАВНЫМИ БУКВАМИ
\end{center}

\vspace{1.1cm}

{
	\noindent\hbox to 0.48\textwidth {%
		\mbox{ } \hfil} %
	\begin{tabular}[t]{l}
		Студент 4 курса очного отделения\\
		Группа 02441--ДБ\\
		Томилов\\ Александр
		Павлович		
	\end{tabular}		
}

\vspace{0.8cm}

{
	\noindent\hbox to 0.48\textwidth {%	
		\mbox{ } \hfil} %
	\begin{tabular}[t]{l}
		Руководитель:\\ к.т.н., доцент\\
		\rule{2.7cm}{0.5pt} Хмельнов А.Е.		
	\end{tabular}		
}

\vspace{0.8cm}

{
	\noindent\hbox to 0.48\textwidth {%	
		\mbox{ } \hfil} %
	\begin{tabular}[t]{l}
		Допущена к защите\\
		\textbf{ВСТАВИТЬ ИМЯ}
%		Зав. кафедрой, д.ф.-м.н., профессор \\
%		\rule{2.7cm}{0.5pt} Пантелеев В.И.
		%		\hfill <<\rule{1.0cm}{0.5pt}>> \rule{3.0cm}{0.5pt} 2016~г.		
	\end{tabular}		
}

\vspace{0.8cm}

\vfill 
\noindent
\begin{minipage}{\textwidth}
	\centering	Иркутск 2017
\end{minipage}
\newpage
%%%----------------------Содержание дипломной работы%%%
\renewcommand{\baselinestretch}{1.5}
\normalsize
%-------------
%Содержание
%-------------
\renewcommand{\contentsname}{СОДЕРЖАНИЕ}
\noindent\tableofcontents

%-------------
%Основная часть
%-------------
\mynonumbersection{ВВЕДЕНИЕ}

Итоговая государственная аттестация является обязательной для всех выпускников. 
Она призвана установить и адекватно оценить уровень подготовки будущих 
специалистов.

Итоговая государственная аттестация осуществляется государственными аттестационными 
комиссиями, основными функциями которых являются комплексная оценка уровня 
подготовки выпускника, решение вопроса о присвоении квалификации и выдача 
выпускнику соответствующего диплома о высшем образовании.

Форма и условия проведения аттестационных испытаний определяется профилирующей кафедрой.

Итоговая государственная аттестация выпускника университета включает в себя
защиту выпускной квалификационной работы в форме дипломной работы или проекта.

\setlength{\itemsep}{0mm}
Выпускная квалификационная работа (ВКР) информатика 
представляет собой законченную разработку (дипломный проект) в профессиональной области, 
в которой: 

%-----------локальное переопределение маркера списка
%Требуется для вывода номера в формате 1) 2) при использовании пунктов списка, начинающихся с 
%строчной буквы
\begin{enumerate}[label=\arabic*)]  
\item сформулирована актуальность и место решаемой задачи 
информационного обеспечения в предметной области;
\item анализируется литература и информация, полученная с помощью 
глобальных сетей, по функционированию подобных систем в данной области 
или в смежных \linebreak[4] предметных областях; 
\item определяются и конкретно описываются выбранные выпускником объемы, 
методы и средства решаемой задачи.
\end{enumerate}

Темы выпускных квалификационных работ определяются профилирующей кафедрой. 
Студенту предоставляется право выбора темы выпускной квалификационной работы 
вплоть до предложения своей тематики с обоснованием целесообразности ее разработки.

При подготовке выпускником квалификационной работы каждому студенту назначается 
руководитель, как правило, из числа преподавателей профилирующей кафедры.

Темы выпускных квалификационных работ и состав руководителей утверждаются приказом 
по университету. Выпускные квалификационные работы подлежат обязательному 
рецензированию. 

К защите выпускной квалификационной работы допускаются лица, завершившие полный курс 
обучения и успешно прошедшие все предшествующие аттестационные испытания, 
предусмотренные учебным планом. Студент, не прошедший в течение установленного 
срока всех испытаний, входящих в состав итоговой государственной аттестации, 
отчисляется из университета и получает академическую справку.



%=======================

\mysection{Название первой главы}
\mysubsection{Название первого параграфа}
Форма и условия проведения аттестационных испытаний определяется профилирующей кафедрой.

Итоговая государственная аттестация выпускника университета включает в себя
защиту выпускной квалификационной работы в форме дипломной работы или проекта.

Выпускная квалификационная работа (ВКР) информатика 
представляет собой законченную разработку (дипломный проект) в профессиональной области, 
в которой: 
\begin{enumerate}[label=\arabic*)]
\item сформулирована актуальность и место решаемой задачи 
информационного обеспечения в предметной области;
\item анализируется литература и информация, полученная с помощью 
глобальных сетей, по функционированию подобных систем в данной области 
или в смежных предметных областях; 
\item определяются и конкретно описываются выбранные выпускником объемы, 
методы и средства решаемой задачи.
\end{enumerate}

Рисунок \ref{first} первый раз встретился в дипломе.
\begin{figure}[h]
\centering
%Рамка только для изображений с белым фоном
%\fbox{\includegraphics[scale=0.7]{img/import.png}}
\caption{Веб-приложение phpMyAdmin}
\label{first}
\end{figure}

\mysubsection{Название второго параграф}

Все помещенные в ВКР иллюстрации (различные схемы, графики, фотографии) именуются рисунками. Размер рисунка не должен превышать принятого для ВКР формата бумаги. Подпись к рисунку размещается непосредственно под ним, выравнивание <<по ширине>>, со стандартным отступом слева. Рисунок помещается сразу после упоминания о нем в тексте. Каждая таблица должна иметь заголовок. Наименование <<Таблица>> с соответствующим номером, помещают над таблицей, используя выравнивание <<по правому краю>>, затем помещают заголовок, используя форматирование <<по центру>>. Сокращения слов в таблице недопустимы. Для всех приведенных в таблице характеристик должны быть указаны единицы измерения и их размерность. Если таблица располагается на нескольких страницах, то каждая последующая страница оформляется определенным образом. Над переносимой частью таблицы, справа пишется <<Продолжение табл.>> или <<Окончание табл.>> и указывается ее номер. При переносе части таблицы на другие страницы название помещают только над первой частью таблицы.

Таблицы и рисунки помещаются в тексте после абзацев, содержащих ссылку на них, а если такой возможности нет, то с первого абзаца на следующей странице. Нумерация таблиц и рисунков сквозная для всей ВКР.

Уравнения и формулы выделяются из текста в отдельную строку. Формула в отдельной строке должна располагаться по центру. Символьные составляющие и числовые коэффициенты формулы расшифровываются. Пояснения значений символов и числовых коэффициентов следует приводить непосредственно под формулой в той же последовательности, в которой они даны в формуле. Первую строку пояснения начинают со слова <<где>> без двоеточия. В конце каждой строки ставят точку с запятой, в конце последней --- точку. В тексте ссылки на формулу даются аналогично ссылкам на таблицу.

Ссылки в тексте делаются следующим образом:
\begin{itemize}[label=--]
	\item на формулу --- формула (2);
\item на рисунок в тексте --- рис. 2;
\item на таблицу --- табл. 3;
\item на приложение --- прил. 1;
\item на стандарты --- (ГОСТ 7.32--2001);
\item на литературу --- [2].
\end{itemize}


%==============
\mysection{Название второй главы}
\mysubsection{Первый параграф}



Темы выпускных квалификационных работ определяются профилирующей кафедрой. 
Студенту предоставляется право выбора темы выпускной квалификационной работы 
вплоть до предложения своей тематики с обоснованием целесообразности ее разработки.

\mysubsubsection{Первый подпараграф}

При подготовке выпускником квалификационной работы каждому студенту назначается 
руководитель, как правило, из числа преподавателей профилирующей кафедры.

Темы выпускных квалификационных работ и состав руководителей утверждаются приказом 
по университету. Выпускные квалификационные работы подлежат обязательному 
рецензированию \cite{Kaz11}. 

\mysubsubsection{Второй подпараграф}

К защите выпускной квалификационной работы допускаются лица, завершившие полный курс 
обучения и успешно прошедшие все предшествующие аттестационные испытания, 
предусмотренные учебным планом. Студент, не прошедший в течение установленного 
срока всех испытаний, входящих в состав итоговой государственной аттестации, не обладающий навыками решения задач на основе таблицы \ref{t1}, 
отчисляется из университета и получает академическую справку.

\begin{table}[h]
\centering
\caption{Таблица FAT}
\label{t1}
\begin{tabular}{r|c|l|}
\cline{2-3}
 8 & 13 & Начало файла D  \\ \cline{2-3}
 7 &  0 &   \\ \cline{2-3}
 6 &  5 &   \\ \cline{2-3}
 5 & -1 &   \\ \cline{2-3}
 4 &  0 & Начало файла B  \\ \cline{2-3}
 3 & -1 &   \\ \cline{2-3}
 2 &  3 &   \\ \cline{2-3}
 1 &  0 &   \\ \cline{2-3}
 0 & 10 &   \\ \cline{2-3}
\end{tabular}
\end{table}

\begin{enumerate}
\item Работа должна показать уровень владения студентом 
знаниями и умениями в выбранной области.

\item Работа должна продемонстрировать готовность студента к самостоятельному
проведению теоретических исследований.

\item В работе необходимо показать умение применять полученные знания для решения
конкретных задач.

\item Работа должна выявлять уровень научного кругозора студента путем указания
места данной работы в теории, ее значимости и актуальности.

\item В работе студент должен продемонстрировать свое логическое мышление, умение
делать выводы из известных результатов.

\item Работа должна демонстрировать умение студента работать с источниками информации
по выбранной теме.
\end{enumerate}
%==========================
\mysection{Третья глава с очень-очень длинным названием}

Иногда в начале главы можно кратко описать ее содержимое. Этот прием применяется том случае, когда глава состоит из нескольких параграфов и имеет значительный объем.
 
\mysubsection{Параграф первый}

В работе \cite{Kotel04} рассматривается ...

В работах \cite{cher, W3Schools:mediatypes, W3Schools:html} приведены алгоритмы ...

Выберите вкладку <<Импорт>>, после чего откроется диалоговое окно с возможностью загрузки импортируемого файла (рисунок~\ref{pic_phpmyadmin_import}).

\begin{figure}[h]
\centering
%Рамка только для изображений с белым фоном
\fbox{\includegraphics[scale=0.7]{img/import.png}}
\caption{Веб-приложение phpMyAdmin}
\label{pic_phpmyadmin_import}
\end{figure}

При возникновении ошибок подключения срабатывает механизм исключений \lstinline{PDOException}. Для того, чтобы отловить исключение, поместите операции с PDO в блок \lstinline{try ... catch}. С помощью метода \lstinline{getMessage()} можно получить сообщение об ошибке, вызвавшей исключение. Пример такого подключения приведен в листинге \ref{php_mysql_connect_ex}.  Чтобы закрыть соединение, нужно уничтожить объект, это можно сделать путем присвоения переменной, которая содержит объект, значения \lstinline{null}. Уничтожение объекта обязательно в том случае, когда после получения данных предполагается достаточно большая их обработка.  

%----!!!Обратите внимание на шапку листинга!!!
\begin{lstlisting}[language=PHP, caption={Устанока соединения с базой данных}, label=php_mysql_connect_ex]
$user = 'root';
$pswd = '';
try {
	$db = new PDO('mysql:host=localhost;dbname=world', $user, $pswd);
	foreach($db->query('SELECT * FROM cities') as $row)
		print_r($row);
	$db = null;
} catch (PDOException $e) {
	print "Error!: " . $e->getMessage() . "<br/>";
	die();
}
\end{lstlisting} 


%-------------
%Заключение 
%-------------
\mynonumbersection{ЗАКЛЮЧЕНИЕ}

В результате выполнения дипломной работы были получены следующие результаты:

\begin{itemize}
\item результат 1; 
\item результат 2;
\item результат 3;
\item результат 4.  
\end{itemize}
    
%-------------
%Список литературы
%-------------
\newpage
%Для включения в диплом списка литературы подключается файл literature.bib
%В нем приведены наиболее часто встречающиеся типы библиографических ссылок
\renewcommand{\refname}{СПИСОК ИСПОЛЬЗОВАННЫХ ИСТОЧНИКОВ}
\addcontentsline{toc}{section}{\refname}
\bibliography{literature}

%-------------
%Приложения
%-------------
\addappendix{Реализация основного модуля}

В приложении можно разместить основные элементы исходного текста программы или большие рисунки.  

\addappendix{Результаты работы программы}

В приложении можно разместить основные элементы исходного текста программы или большие рисунки.  

\end{document}

